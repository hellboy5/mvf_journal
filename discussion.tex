%%\vspace{-0.7325cm}
\section{Conclusion}
%\vspace{-0.325cm}
We have presented a novel mid-level representation of an image as a set of Medial Visual Fragments, some of which delineate objects and their parts. The approach generates these proposals by simultaneously representing alternative and conflicting grouping  options. The approach is tractable due to the use of a medial representation of proposals which uses multiple cues simultaneously and due to the use of a containment graph which avoids duplication and which only explores viable options. It is clearly shown above that the fragments generated by our approach contain full object segments, matching or exceeding the current methods, although the primary focus is on producing fragments that represent object parts effectively. The framework has a great deal of unexplored potential, \textit{(i) } more effective strategies can be used to search the space, and \textit{(ii)} not all the likelihood functions for individual transforms use the full set of cues available. We plan to explore recognition strategies based on these fragments. These developments are expected to significantly increase the performance of this approach.
 

% The distribution of fragments in the PR domain shows generally focused
% distribution near the top right corner, with some outliers. These outliers
% are frequently images with a significant degree of structure texture (weaved basket). As a bottom-up process the algorithm cannot decide whether the
% individual strands of the basket are significant or should be ignored.
% The tremendous number of these low-level fragments overwhelm the construction
% process so that the coarser fragments may not be formed, hence the low
% precision-recall for these images.
% Clearly, we should capture structure texture in the process.
%  
% Explain that we have trouble in highly textured areas, especially with
% regular texture.

%%\vspace{-0.325cm} 

